\documentclass[11pt]{article}

\usepackage[margin=1in]{geometry}
\usepackage{amsmath, amssymb, amsthm}
\usepackage{hyperref}
\usepackage{enumitem}
\usepackage[normalem]{ulem} % for strikethrough

\hypersetup{
  colorlinks=true,
  linkcolor=blue,
  citecolor=blue,
  urlcolor=blue
}

\title{Review of ``Evaluating Health Policy Without a Fixed Lifespan: A Steady-State Foundation for Quality, Maintenance, and Risk''}
\author{Reviewer Report}
\date{\today}

\begin{document}
\maketitle

\section*{Executive Summary}

This paper proposes a theoretical framework for evaluating health policies in a hypothetical world without fixed biological lifespan limits. The core contribution is replacing traditional lifetime-summed QALYs with a \emph{steady QALY flow} metric based on renewal-reward theory. The framework is mathematically coherent within its stated assumptions, but several foundational issues merit careful consideration before the framework could be applied.

\textbf{Overall Assessment:} The paper presents a mathematically sound but philosophically and practically limited foundation. The renewal-reward approach is clever and well-executed within its assumptions, but those assumptions carry significant implicit normative content that partially undermines the paper's stated goal of explicit normativity.

\textbf{Note:} Following the initial review, all mathematical issues identified have been addressed in the revised document. See Section~\ref{sec:amendments} for a complete record of amendments.

\section{Summary of the Paper's Contributions}

The paper makes the following claims:
\begin{enumerate}[label=(\alph*)]
    \item Traditional lifetime-summed health metrics fail when lifespans are unbounded.
    \item A regenerative (renewal) framework with QALY flow $J(\pi) = \mathbb{E}[R^\pi]/\mathbb{E}[\tau^\pi]$ resolves this.
    \item The framework admits well-defined optimization, sensitivity analysis, and extensions for irreversibility and risk.
    \item All normative choices are made explicit through axioms and parameters.
\end{enumerate}

\section{Evaluation of Mathematical Rigor}

\subsection{Strengths}

\subsubsection{Renewal-Reward Foundation (Theorem 3.1)}
The core mathematical result---that $Q(T)/T \to J(\pi)$ almost surely---is a direct application of the classical renewal-reward theorem. The setup is correct:
\begin{itemize}
    \item The regenerative assumption (Assumption 2) creates i.i.d.\ cycles.
    \item Boundedness of $q(\cdot) \in [0,1]$ ensures $0 \le R^\pi \le \tau^\pi$.
    \item The condition $\mathbb{E}[\tau^\pi] \in (0,\infty)$ is necessary and sufficient for the standard renewal-reward theorem.
\end{itemize}
\textbf{Verdict:} Mathematically sound, assuming the renewal-reward theorem's conditions are met.

\subsubsection{Boundedness Results (Propositions 3.1 and 3.2)}
The proofs that $J(\pi) \in [0,1]$ are trivially correct given $0 \le R^\pi \le \tau^\pi$ a.s.

\subsubsection{Lipschitz Stability (Theorem 5.1)}
The stability result is correctly derived. The key insight that sensitivity of the ratio $a/b$ is controlled by the lower bound on $b$ and Lipschitz constants on $a$ and $b$ separately is standard but correctly applied.

\subsection{Weaknesses and Gaps}

\subsubsection{Missing Proof of Theorem 4.1}
\textbf{[RESOLVED]} The existence theorem for optimal stationary policies in the finite model (Theorem 4.1) now includes a complete proof. The proof correctly observes that:
\begin{enumerate}
    \item The policy space is finite (finite $\mathcal{S}$, finite $\mathcal{A}(x)$).
    \item $W_\lambda(f)$ is well-defined for each policy by Assumption 3.
    \item The maximum over a finite set of finite values is attained.
\end{enumerate}

\subsubsection{Non-Explosiveness (Assumption 1)}
\textbf{[RESOLVED]} The revised document now includes a remark providing sufficient conditions for non-explosiveness:
\begin{itemize}
    \item Uniformly bounded transition rates: $\sup_{x,a} Q(\mathcal{X}\setminus\{x\} | x, a) < \infty$.
    \item Foster--Lyapunov criteria for countable state spaces with appropriate Lyapunov function.
\end{itemize}
This adequately addresses the concern.

\subsubsection{Truncated Flow Definition (Definition 3.2)}
\textbf{[RESOLVED]} The revised document now includes:
\begin{enumerate}
    \item Proposition 3.2 (Well-posedness): proves $\mathbb{E}[\tau_T^\pi] > 0$ for all $T > 0$.
    \item Proposition 3.4 (Convergence): proves that for stationary Markov policies on finite state spaces with $\dagger$ reachable from every state, $J^+ = J^-$ and equals the standard $J(\pi)$.
\end{enumerate}
This adequately characterizes when the truncated flow is well-defined and convergent.

\subsubsection{Theorem 5.1: Condition Verification}
\textbf{[RESOLVED]} The revised document now includes a remark providing sufficient conditions:
\begin{itemize}
    \item For Assumption 3: uniform lower bound on catastrophe rate from reachable states.
    \item For Assumption 4: Lipschitz continuity of transition rates implies Lipschitz continuity of expectations via Dynkin's formula or coupling arguments.
\end{itemize}

\subsubsection{Measurability and Regularity}
The paper assumes measurability of $q: \mathcal{X} \to [0,1]$ and $c: \mathcal{S} \times \mathcal{A} \to [0, \infty)$ but never specifies the $\sigma$-algebras on $\mathcal{S}$, $\mathcal{A}$, or $\mathcal{X}$. For general state spaces, this is essential for the integrals in (3)--(5) to be well-defined.

\section{Evaluation of Assumptions}

\subsection{Axiom Analysis}

\subsubsection{Axiom 1 (Well-posedness)}
\textbf{Claim:} Policy evaluation must be finite and support stable marginal comparisons.

\textbf{Critique:} This axiom is definitional---it describes what a good metric should do, not what the world is like. It is satisfied by construction once the renewal-reward framework is adopted.

\subsubsection{Axiom 2 (Horizon Invariance)}
\textbf{Claim:} The criterion should not depend on an exogenous terminal time.

\textbf{Critique:} \textbf{[ACKNOWLEDGED]} The revised document now explicitly notes that Axioms 2--4 ``reflect welfarist and experientialist commitments that, while standard in health economics, are substantive ethical positions rather than purely structural requirements.'' This transparency is appropriate and addresses the concern that normative choices were presented as structural.

\subsubsection{Axiom 3 (Experienced Welfare as Flow)}
\textbf{Claim:} Health value accrues through instantaneous quality $q(t)$.

\textbf{Critique:} \textbf{[ACKNOWLEDGED]} This commits to an experientialist account of welfare, ruling out preference satisfaction or narrative accounts of wellbeing. The revised document now explicitly acknowledges this as a substantive ethical position inherited from standard health economics.

\subsubsection{Axiom 4 (Death as Truncation)}
\textbf{Claim:} Death matters only because it ends future experience; no independent terminal disvalue.

\textbf{Critique:} \textbf{[ACKNOWLEDGED]} This is contested in the philosophy of death (e.g., desire-frustration theory assigns intrinsic badness to death). The revised document now explicitly acknowledges this as a normative position rather than a structural requirement.

\subsubsection{Axiom 5 (Explicit Normativity)}
\textbf{Claim:} All normative choices must be explicit parameters.

\textbf{Critique:} The paper does not fully achieve this:
\begin{itemize}
    \item The choice of ratio (expected reward / expected time) vs.\ expected ratio $\mathbb{E}[R^\pi / \tau^\pi]$ is implicit.
    \item The regeneration assumption (replacement upon death) is a massive normative choice that is stated technically rather than ethically defended.
    \item Risk neutrality in the base framework is implicit (variance is added as an extension).
\end{itemize}

\subsubsection{Axiom 6 (Person-Time Interpretation)}
\textbf{Claim:} Public health should optimize welfare per unit population-time.

\textbf{Critique:} \textbf{[ACKNOWLEDGED]} This implies fungibility between individuals. The revised document adds a remark noting that this fungibility is ``inherited from standard QALY-based health economics'' and argues that unbounded lifespans make it \emph{more} defensible due to trajectory convergence. This is a reasonable response---the framework is no more committed to fungibility than existing practice.

\subsection{Key Assumption: Regeneration (Assumption 2)}

This is the load-bearing assumption of the entire framework.

\textbf{What it says:} Upon death, a new individual immediately replaces the old one with i.i.d.\ initial state, and the policy continues.

\textbf{Original concerns and responses:}

\begin{enumerate}
    \item \textbf{Interpersonal aggregation smuggled in:} \textbf{[ACKNOWLEDGED]} The revised document now includes a remark explicitly noting that fungibility is inherited from standard health economics, not introduced by this framework.

    \item \textbf{Immediate replacement:} \textbf{[ADDRESSED]} The revised document adds a ``Large-population interpretation'' remark explaining that immediate replacement reflects continuous demographic flow in a large population, not literal individual replacement.

    \item \textbf{Identical initial distribution:} \textbf{[ADDRESSED]} The large-population interpretation frames this as modeling the steady-state regime rather than an empirical claim about entrant heterogeneity.

    \item \textbf{Independence:} \textbf{[ADDRESSED]} The revised document notes that independence follows from population size: any single death is statistically uncorrelated with the vast pool of existing individuals.
\end{enumerate}

The large-population reframing is persuasive: the regenerative structure is a \emph{modeling consequence} of targeting cross-sectional welfare in a stationary population, not an empirical claim about individual-level replacement dynamics.

\subsection{Implicit Assumptions Not Stated}

\subsubsection{Risk Neutrality}
\textbf{[ADDRESSED]} The revised document now includes a remark after Definition 3.1 explicitly stating that the base framework embeds risk neutrality. The remark clarifies that risk aversion can be addressed via explicit extensions (variance penalty, CVaR), making risk preferences an explicit parameter rather than a hidden assumption.

\subsubsection{Choice of Ratio}
The paper uses $\mathbb{E}[R]/\mathbb{E}[\tau]$ (ratio of expectations) rather than $\mathbb{E}[R/\tau]$ (expectation of ratio). These are not equal in general. The former is justified by the renewal-reward theorem; the latter would weight short cycles more heavily. This choice is implicit.

\subsubsection{No Catastrophe Aversion}
The framework treats catastrophe (death) purely as loss of future quality-time. There is no aversion to catastrophe \emph{per se} beyond this. This means:
\begin{itemize}
    \item A policy with 10\% annual death rate but high quality when alive may score the same as a policy with 1\% death rate and slightly lower quality.
\end{itemize}
Many ethical frameworks would add explicit disvalue to catastrophe events.

\section{Technical Issues and Recommendations}

\subsection{Theorem 5.1: Proof}
\textbf{[RESOLVED]} The proof of Theorem 5.1 has been revised to use the correct algebraic identity:
$$\frac{a}{b} - \frac{a'}{b'} = \frac{(a-a')b' + a'(b'-b)}{bb'}$$
The derivation is now fully rigorous.

\subsection{Definition 3.2: Well-Posedness}

For the truncated flow, the paper should verify that $\mathbb{E}[\tau_T^\pi] > 0$ for all $T > 0$. This holds if $\mathbb{P}(\tau > 0) > 0$, which is true if there is a positive probability of surviving any positive time. This should be stated explicitly.

\subsection{Section 6.2: Volatility Measure}
\textbf{[RESOLVED]} The revised document now:
\begin{enumerate}
    \item Introduces Assumption 5 requiring a minimum cycle length $\epsilon > 0$ for the variance-based risk measure.
    \item Provides an alternative CVaR-based risk measure applied to $R^\pi$ directly, which is always well-defined.
\end{enumerate}

\subsection{Appendix: Cross-Sectional Interpretation}
\textbf{[RESOLVED]} The revised appendix now:
\begin{enumerate}
    \item States the result as a formal proposition (Proposition A.1).
    \item Explicitly requires $\mathbb{E}[\tau^\pi] \in (0, \infty)$ as a hypothesis.
    \item Provides a proof sketch referencing the key renewal theorem.
\end{enumerate}

\section{Conceptual and Philosophical Issues}

\subsection{The ``Unbounded Lifespan'' Premise}

The paper's motivation is a world without fixed biological lifespan. However:
\begin{enumerate}
    \item This is a hypothetical that does not match current reality or near-term projections.
    \item Even in such a world, discounting may be justified by uncertainty about future preferences, technology, or catastrophic risk at civilization level.
    \item The framework assumes ongoing institutional stability and policy continuity---a strong assumption over unbounded horizons.
\end{enumerate}

\subsection{Person-Time vs.\ Person Welfare}

The framework optimizes welfare per person-time, which has counterintuitive implications:
\begin{itemize}
    \item A policy that extends the lives of healthy people contributes no more per unit time than one that maintains the status quo.
    \item Life extension is valuable only if it increases the quality-weighted survival rate, not intrinsically.
\end{itemize}
This may conflict with common intuitions that saving a life has intrinsic value.

\subsection{Comparison with Discounting}

The paper criticizes discounting but does not engage with the full range of justifications:
\begin{itemize}
    \item \textbf{Uncertainty:} Future benefits may not materialize due to technological, political, or catastrophic changes.
    \item \textbf{Opportunity cost:} Resources invested now could grow, making future health cheaper.
    \item \textbf{Intergenerational equity:} Future generations may be wealthier, justifying less present sacrifice.
\end{itemize}
The steady-flow framework avoids discounting by assuming perfect regeneration and stationarity, which implicitly assumes these concerns away.

\section{Summary of Issues}

\subsection{Mathematical Issues}
\textit{Note: The mathematical issues listed below have been addressed in the revised document. See Section~\ref{sec:amendments} for details.}

\begin{enumerate}
    \item \sout{Theorem 4.1 needs proof (minor).} \textbf{[RESOLVED]}
    \item \sout{Non-explosiveness conditions need to be specified for general models.} \textbf{[RESOLVED]}
    \item \sout{Convergence of $J^{\pm}(\pi)$ needs characterization.} \textbf{[RESOLVED]}
    \item \sout{Lipschitz assumptions (Assumptions 3--4) need verification conditions.} \textbf{[RESOLVED]}
    \item \sout{Volatility measure may be ill-defined when cycle lengths can be small.} \textbf{[RESOLVED]}
\end{enumerate}

\subsection{Assumption Issues}
\begin{enumerate}
    \item \sout{Regeneration assumption embeds strong interpersonal aggregation claims.} \textbf{[ACKNOWLEDGED---inherited from standard health economics]}
    \item \sout{Axioms 2--4 are substantive ethical positions, not structural requirements.} \textbf{[ACKNOWLEDGED in revised text]}
    \item Risk neutrality is implicit in the base framework. (Noted but acceptable given explicit extensions.)
    \item \sout{Immediate, identical, independent replacement is unrealistic.} \textbf{[ADDRESSED via large-population interpretation]}
\end{enumerate}

\subsection{Missing Elements}
\begin{enumerate}
    \item No comparison with existing frameworks (discounted QALYs, DALYs, healthy life expectancy).
    \item No empirical application or calibration.
    \item No discussion of how to estimate the parameters (quality function, transition rates) in practice.
    \item No engagement with the extensive literature on social welfare functions and interpersonal comparisons.
\end{enumerate}

\section{Conclusion}

This paper makes a coherent mathematical contribution: given a regenerative process interpretation of health policy, renewal-reward theory provides a well-founded objective function. The technical execution is now complete, with all proofs and conditions properly specified.

Following revisions, the paper's claim to ``explicit normativity'' is now substantially realized. The axiom section acknowledges that Axioms 2--4 encode welfarist and experientialist commitments. The large-population interpretation reframes the regeneration assumption as a modeling consequence rather than an empirical claim. The remark on interpersonal aggregation correctly notes that fungibility is inherited from standard health economics.

The framework is best understood as a natural extension of QALY-based health economics to unbounded lifespans, suitable for settings where:
\begin{itemize}
    \item Interpersonal fungibility of welfare is accepted (as in standard practice).
    \item The policy-relevant unit is cross-sectional person-time under a stationary regime.
    \item Risk neutrality across cycles is appropriate (or explicit risk extensions are used).
    \item The population is large enough that the regenerative abstraction holds.
\end{itemize}

\subsection*{Recommendation}

The revised paper now presents a mathematically complete and conceptually transparent framework. The acknowledgment of normative commitments and the large-population interpretation successfully address the main concerns from the original review. The paper makes a solid contribution to the theoretical foundations of health policy evaluation under unbounded lifespans.

\section{Record of Amendments}\label{sec:amendments}

The following mathematical issues were identified in the original review and have been resolved in the revised document (dated \today):

\subsection{Amendment 1: Proof of Theorem 4.1}
\textbf{Original issue:} Theorem 4.1 (existence of optimal stationary policy in finite model) was stated without proof.

\textbf{Resolution:} A complete proof has been added. The proof establishes that the policy space is finite, $W_\lambda(f)$ is well-defined for each policy, and the maximum over a finite set is attained.

\subsection{Amendment 2: Non-Explosiveness Conditions}
\textbf{Original issue:} The non-explosiveness assumption was stated but no sufficient conditions were provided.

\textbf{Resolution:} A remark has been added after Assumption 1 providing:
\begin{itemize}
    \item Sufficient condition via uniformly bounded transition rates.
    \item Foster--Lyapunov criteria for countable state spaces.
\end{itemize}

\subsection{Amendment 3: Truncated Flow Convergence}
\textbf{Original issue:} No characterization was given for when $J^+ = J^-$.

\textbf{Resolution:} Two propositions have been added:
\begin{itemize}
    \item Proposition 3.2: Well-posedness ($\mathbb{E}[\tau_T^\pi] > 0$ for all $T > 0$).
    \item Proposition 3.4: Convergence ($J^+ = J^-$ for finite state spaces with reachable absorbing state).
\end{itemize}

\subsection{Amendment 4: Sufficient Conditions for Stability Assumptions}
\textbf{Original issue:} Assumptions 3--4 (uniform lower bound on expected cycle length, Lipschitz continuity) were stated abstractly without verification conditions.

\textbf{Resolution:} A remark has been added providing sufficient conditions:
\begin{itemize}
    \item Lower bound on catastrophe rate implies bounded expected cycle length.
    \item Lipschitz transition rates imply Lipschitz expectations via Dynkin's formula/coupling.
\end{itemize}

\subsection{Amendment 5: Theorem 5.1 Proof Precision}
\textbf{Original issue:} The proof had a minor imprecision in the algebraic derivation.

\textbf{Resolution:} The proof has been rewritten using the explicit identity $\frac{a}{b} - \frac{a'}{b'} = \frac{(a-a')b' + a'(b'-b)}{bb'}$ with clear steps.

\subsection{Amendment 6: Volatility Measure Well-Definedness}
\textbf{Original issue:} The variance-based risk measure $\text{Var}(R^\pi/\tau^\pi)$ may be undefined when cycle lengths can be arbitrarily small.

\textbf{Resolution:} 
\begin{itemize}
    \item Assumption 5 (minimum cycle length) has been added for the variance-based measure.
    \item An alternative CVaR measure applied to $R^\pi$ (always well-defined) has been provided.
\end{itemize}

\subsection{Amendment 7: Appendix Formalization}
\textbf{Original issue:} The cross-sectional interpretation was stated informally without explicit conditions.

\textbf{Resolution:} The result is now stated as a formal proposition with:
\begin{itemize}
    \item Explicit hypothesis: $\mathbb{E}[\tau^\pi] \in (0, \infty)$.
    \item Proof sketch referencing the key renewal theorem.
\end{itemize}

\subsection{Amendment 8: Normative Transparency and Large-Population Interpretation}
\textbf{Original issue:} The review identified that Axioms 2--4 embed substantive ethical positions presented as structural, that interpersonal fungibility was implicit, and that the regeneration assumption (immediate, identical, independent replacement) was unrealistic.

\textbf{Resolution:} Three additions address these concerns:
\begin{itemize}
    \item The axiom section now explicitly acknowledges that Axioms 2--4 ``reflect welfarist and experientialist commitments that, while standard in health economics, are substantive ethical positions.''
    \item A new remark on interpersonal aggregation notes that fungibility is inherited from standard QALY-based health economics and argues that unbounded lifespans make it more defensible due to trajectory convergence.
    \item A ``Large-population interpretation'' remark reframes the regeneration assumption as modeling cross-sectional welfare in a stationary population, where:
    \begin{itemize}
        \item Immediate replacement reflects continuous demographic flow.
        \item Independence follows from large population size.
        \item Identical initial distribution reflects the steady-state modeling target.
    \end{itemize}
\end{itemize}
This reframing successfully transforms questionable empirical claims into defensible modeling choices.

\subsection{Amendment 9: Explicit Risk Neutrality in Base Framework}
\textbf{Original issue:} Risk neutrality was implicit in the base framework. The objective $J(\pi)$ depends only on expected values, not on variance or volatility, but this design choice was not made explicit.

\textbf{Resolution:} A remark has been added after Definition 3.1 (Steady QALY flow) that:
\begin{itemize}
    \item Explicitly states that $J(\pi)$ depends only on expected values, not on higher moments.
    \item Explains that this embeds a risk-neutral preference: two policies with identical $J(\pi)$ but different cycle-to-cycle volatility are treated as equally desirable.
    \item Notes that risk aversion can be addressed via explicit extensions (variance penalty, CVaR).
    \item Emphasizes that the base framework is risk-neutral by design, making risk preferences an explicit parameter.
\end{itemize}
This brings risk neutrality in line with the principle of Axiom~\ref{ax:explicit} (explicit normativity).

\vspace{1em}
\noindent\textbf{Status:} All major issues identified in the original review have now been addressed. The mathematical gaps have been filled, and the conceptual concerns regarding normative transparency have been resolved through explicit acknowledgment and reframing. The framework is now more transparent about its commitments and provides a coherent foundation for health policy evaluation under unbounded lifespans.

\end{document}
