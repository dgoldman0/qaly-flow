\documentclass[11pt]{article}

\usepackage[margin=1in]{geometry}
\usepackage{setspace}
\usepackage{hyperref}
\usepackage{enumitem}
\usepackage[round]{natbib}
\usepackage[T1]{fontenc}
\usepackage{orcidlink}

\hypersetup{
  colorlinks=true,
  linkcolor=blue,
  citecolor=blue,
  urlcolor=blue
}

\title{Healthy Ageing in a World Where Ageing Might Become Manageable:\\
A Public Health Framing for Long-Horizon Decision Making}
\author{Daniel S. Goldman \orcidlink{0000-0003-2835-3521}}
\date{\today}

\begin{document}
\maketitle
\doublespacing

\begin{abstract}
Biomedical advances increasingly aim to delay, slow, or partially reverse biological processes associated with ageing. Even without assuming ``immortality,'' this possibility challenges a quiet assumption embedded in much of population health evaluation: that lives and benefits are naturally bounded in time. As horizons lengthen, familiar ways of describing value as a lifetime total can become difficult to interpret and may motivate unstable priorities. This paper offers a public-health translation of a companion technical framework that was developed for long horizons. The translation avoids mathematics and focuses on language that public health scientists can use: durable streams of healthy functioning, maintenance burden, catastrophic risk management, deliverability at scale, and equity. We connect these ideas to the UN Decade of Healthy Ageing and to integrated care and multimorbidity evidence, and we outline a research and policy agenda that stays grounded in health systems and lived experience.
\end{abstract}

\section*{Plain-language summary}
Public health already manages many problems that are not ``one and done.'' We vaccinate, monitor, screen, support behavior change, and build care systems that keep working year after year. If future medicine makes ageing more manageable, the public-health question becomes: \emph{how do we sustain healthy functioning for a very long time, fairly, at population scale?} The key shift in this paper is conceptual. Instead of adding up a lifetime total of benefits and costs, we focus on the long-run pattern of health that an intervention helps generate and sustain, along with the system capacity required to keep that pattern going.

\section{Why this topic matters now}
Healthy ageing is already a central global public-health priority \citep{WHO2020DecadePlan}. At the same time, biomedicine is increasingly organized around delaying or modifying underlying ageing processes, not only treating one disease at a time. Whether such approaches deliver modest improvements or larger discontinuities, they raise a planning challenge: as horizons lengthen, the dominant questions become about maintenance, repeatability, system capacity, and fair access.

Public health is uniquely positioned to lead this conversation because it can:
\begin{itemize}[leftmargin=*]
  \item insist that success means functional ability and lived experience, not only survival;
  \item translate biomedical advances into delivery strategies that can reach populations;
  \item anticipate second-order effects (workforce, financing, inequities, infrastructure strain) that emerge over decades.
\end{itemize}

\section{The hidden assumption in many evaluations: a bounded horizon}
Many evaluation habits implicitly assume that time is naturally bounded: benefits accrue over a life course that ends ``soon enough'' for lifetime summaries to behave as expected. The companion technical framework motivating this paper shows why this assumption becomes fragile when:
\begin{enumerate}[leftmargin=*]
  \item interventions can be repeated or maintained (for example, ongoing therapy, periodic monitoring, or repeated treatment cycles);
  \item some harms or failures are sudden and irreversible (for example, abrupt transitions into severe disability, or low-probability adverse events with high severity);
  \item survival horizons shift, making finite lifetime totals difficult to compare across policies.
\end{enumerate}

The point is not that existing tools are useless. The point is that long horizons change what matters most: the ability to reliably generate and sustain healthy functioning over time, under uncertainty, with equitable access.

\section{A public-health translation: from lifetime totals to durable streams}
We propose language that mirrors the technical framework without mathematics.

\subsection{Core idea}
Instead of asking: \emph{``How much total benefit does this produce over a lifetime?''} ask:
\begin{quote}
\emph{``What long-run pattern of healthy functioning does this help produce and sustain, and what does it require from systems to keep that pattern going?''}
\end{quote}

This translation emphasizes three practical dimensions.

\paragraph{(1) Maintenance burden}
How much ongoing effort is required to keep benefits flowing: clinical visits, monitoring, adherence support, supply chains, workforce time, and the social supports that enable people to follow care plans? For long horizons, small recurring burdens can dominate.

\paragraph{(2) Catastrophic risk management}
How does the intervention change the chance of abrupt, high-severity transitions? Long horizons magnify the importance of low-probability, high-impact risks, including adverse events, missed follow-up, discontinuities in coverage, and failures of coordination.

\paragraph{(3) Deliverability at scale}
Can the health system deliver the intervention as a repeating service over years or decades? If delivery requires specialized centers, intensive monitoring, or frequent re-treatment, its long-run value depends on whether capacity can be built, financed, and sustained.

\subsection{Why this is public health, not abstract futurism}
Public health already thinks in streams:
\begin{itemize}[leftmargin=*]
  \item surveillance produces continuous information;
  \item primary care manages conditions through ongoing relationships;
  \item prevention programs work through sustained environments and policies.
\end{itemize}
The UN Decade of Healthy Ageing adopts long time horizons and multi-sector implementation \citep{WHO2020DecadePlan}. A long-horizon framing extends that agenda: it makes sustained function, continuity, and system capacity the default lens for evaluation and planning.

\section{Connecting to healthy ageing practice: integrated care as the delivery backbone}
A long-horizon view naturally centers integrated care. The WHO ICOPE handbook provides practical guidance for delivering older-person care as a coordinated continuum, typically rooted in primary care and iterative follow-up \citep{WHO2024ICOPE}. This matters even more if ageing-modifying interventions arrive, because delivery will still require:
\begin{itemize}[leftmargin=*]
  \item person-centred assessment tied to functional goals;
  \item multidisciplinary coordination and clear role definitions;
  \item monitoring and adjustment over time;
  \item linkage to community supports and long-term care when needed.
\end{itemize}

Evidence syntheses on multimorbidity and integrated care highlight both promise and heterogeneity. A scoping review of systematic reviews shows wide variation in what integrated care includes and how it is evaluated, reinforcing the need for measurement aligned with sustained function rather than short-term endpoints alone \citep{Rohwer2023IntegratedCareScoping}. A recent review of care models for chronic multimorbidity emphasizes that durable outcomes depend on organizational arrangements such as leadership, financing, workforce design, and information systems, and that meaningful models are shaped by context and implementation choices \citep{Endalamaw2024MultimorbidityCareModels}. These findings support a central claim of the long-horizon framework: value depends not only on what an intervention can do biologically, but also on whether systems can deliver it continuously and safely.

\section{Policy implications under ``ageing-as-manageable'' scenarios}
Even modest shifts in the pace of ageing can alter population health needs over decades. The long-horizon lens suggests several policy implications that are actionable now.

\subsection{From episodes to infrastructures}
Interventions that require repeat delivery become infrastructures. This changes the policy question from ``Should we cover this once?'' to ``Can we sustain this service reliably and fairly over time?'' That shift makes the following system features central: continuity of coverage, workforce planning, data systems for monitoring and follow-up, and stable supply chains.

\subsection{A prevention--care continuum, not a binary}
Public health debates can slip into a false choice between prevention and care. Long-horizon conditions dissolve the boundary: maintaining function for decades requires both prevention (risk reduction, supportive environments) and care (continuous management, rehabilitation, assistance when decline occurs). ICOPE’s continuum provides a practical model of this blended approach \citep{WHO2024ICOPE}.

\subsection{Equity becomes structural, not incidental}
If benefits accrue through long-run access, inequity can compound. Priority setting frameworks that explicitly incorporate fairness concerns become more salient. Work on equity-sensitive priority setting highlights how society may value similar health gains differently depending on severity and unmet lifetime health, and how fairness objectives can be made explicit in allocation decisions \citep{Mahdiani2024ProportionalShortfall}. In a long-horizon setting, equity questions expand to include:
\begin{itemize}[leftmargin=*]
  \item equitable access to maintenance therapies, monitoring, and supportive environments;
  \item prevention of longevity stratification by income, geography, or race;
  \item governance of pricing and financing for repeating interventions.
\end{itemize}

\subsection{A grounded endpoint: health span and functional ability}
Public-health audiences often resist claims of radical life extension. A constructive alternative is to target health span and functional ability: success is measured by healthy years and what people can do, not survival alone \citep{Olshansky2022HealthSpan}. This aligns with healthy ageing goals and keeps the discussion connected to lived outcomes.

\section{A research agenda for public health}
We outline a practical agenda designed to be useful regardless of how quickly ageing-modifying interventions mature.

\subsection{Measurement and surveillance}
Develop indicators that track durable function and lived capability, paired with system indicators that determine whether benefits can be sustained: continuity, access, affordability, and workforce capacity. Healthy ageing frameworks already emphasize these domains \citep{WHO2020DecadePlan}.

\subsection{Evaluation for long-run maintenance}
Complement trials with long-term follow-up, pragmatic evaluations, and real-world evidence. Evaluate not only whether an intervention works in ideal conditions, but whether it can be delivered continuously without creating unsustainable burdens or widening inequities. For repeating interventions, key outcomes include persistence of benefit, discontinuation rates, monitoring intensity, and failure modes.

\subsection{Implementation science for integrated care}
Integrated care models vary widely \citep{Rohwer2023IntegratedCareScoping}. Public health can accelerate learning by standardizing descriptions of intervention components, documenting context, and building implementation research that focuses on scalability and sustainability. Lessons from multimorbidity care models emphasize the importance of financing, information systems, and workforce organization \citep{Endalamaw2024MultimorbidityCareModels}.

\subsection{Governance and social legitimacy}
Long-horizon interventions raise questions about intergenerational fairness, priority setting, and public trust. Equity-focused approaches provide vocabulary and structure for these debates \citep{Mahdiani2024ProportionalShortfall}. Public health institutions should anticipate and engage these issues early, before policy becomes reactive to hype or backlash.

\section{Relationship to the companion technical framework}
This manuscript is a translation of a companion technical note \citep{Framework2026}. The technical work formalizes the idea that when horizons expand, value should be understood as the long-run production and maintenance of health under uncertainty, and that catastrophic risks and irreversibilities matter. Here we use that logic to motivate a public-health paper that adopts healthy ageing and functional ability as the public-facing endpoint, treats integrated care and multimorbidity care models as the real-world implementation layer, and centers equity and governance as first-order issues.

\section{Conclusion}
A future where ageing becomes more manageable would not eliminate public health; it would deepen its relevance. The central challenge would shift from extending life to sustaining healthy functioning fairly over long horizons. Doing that requires language and evaluation approaches that emphasize durable streams of health, maintenance burden, catastrophic risk management, deliverability, and equity. The healthy ageing agenda already provides a platform for this work. The next step is to align research, measurement, and policy design with the realities of long-horizon maintenance.

\newpage
\bibliographystyle{plainnat}
\bibliography{references}

\end{document}
