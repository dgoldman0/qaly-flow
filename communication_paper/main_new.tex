\documentclass[11pt]{article}

\usepackage[margin=1in]{geometry}
\usepackage{setspace}
\usepackage{hyperref}
\usepackage{enumitem}
\usepackage[round]{natbib}
\usepackage[T1]{fontenc}
\usepackage{orcidlink}
\usepackage{booktabs}

\hypersetup{
  colorlinks=true,
  linkcolor=blue,
  citecolor=blue,
  urlcolor=blue
}

\title{Sustaining Health Over Long Horizons:\\
Integrated Care, Maintenance Burden, and Equity When Ageing Becomes Manageable}
\author{Daniel S. Goldman \orcidlink{0000-0003-2835-3521}}
\date{\today}

\begin{document}
\maketitle
\doublespacing

\begin{abstract}
Biomedical advances increasingly target the underlying mechanisms of ageing rather than individual diseases. Even modest success would shift the central challenge of population health from curing discrete conditions to sustaining functional ability over extended horizons. This paper applies the QALY flow framework---which evaluates policy by the long-run rate of health production per unit person-time---to three practical domains: integrated care delivery, maintenance burden assessment, and equity in sustained access. We draw on the UN Decade of Healthy Ageing, the WHO ICOPE implementation guidance, systematic evidence on multimorbidity care models, the health span literature, and equity-focused priority setting frameworks. The result is a research and policy agenda grounded in health systems, delivery science, and existing institutional commitments.
\end{abstract}

\section*{Key messages}

When health can be maintained rather than merely restored, the policy question shifts from ``How much total benefit?'' to ``What rate of healthy functioning can we sustain, and at what cost?'' QALY flow---average quality per unit person-time---remains bounded and interpretable as horizons extend, providing a stable basis for long-run planning. Under this framework, maintenance burden, catastrophic risk management, and deliverability at scale become dominant evaluation criteria rather than secondary considerations. Equity concerns take on new urgency because sustained access gaps accumulate into widening health divides over time; those excluded from maintenance regimes fall progressively further behind those with continuous access. The policy infrastructure for this transition already exists: the UN Decade of Healthy Ageing provides the political platform and the ICOPE implementation guidance provides the delivery architecture.

\section{Introduction}

The UN Decade of Healthy Ageing (2021--2030) organizes global action around a central insight: healthy ageing means enabling people to ``do what they value'' through sustained functional ability \citep{WHO2020DecadePlan}. The Decade's four action areas---changing attitudes toward ageing, fostering age-friendly environments, delivering person-centred integrated care, and ensuring access to long-term care---all emphasize continuity over time rather than episodic intervention. This framing already moves public health toward maintenance thinking.

Biomedical research is increasingly organized around the same intuition. Therapies targeting senescence, inflammation, metabolic dysfunction, and other ageing mechanisms aim to slow or partially reverse biological decline rather than cure isolated diseases \citep{Olshansky2022HealthSpan}. If these approaches succeed even modestly, they would extend the period during which individuals require ongoing health support. The public health challenge becomes sustaining functional ability over very long periods---potentially decades longer than current planning horizons assume.

The QALY flow framework provides the evaluative structure for this transition \citep{Framework2026}. Where traditional health technology assessment sums quality-adjusted life years over a bounded lifetime, the flow framework defines value as the long-run rate of health production per unit person-time under a continuing regime. This ratio remains bounded and interpretable regardless of how long individuals live. It naturally foregrounds the questions that matter for extended horizons: Can the intervention be delivered continuously? What ongoing resources does it require? Who has sustained access?

This paper translates the flow framework into public health practice. We apply it to three domains: integrated care as the delivery backbone (Section 2), maintenance burden as a policy variable (Section 3), and equity as a structural concern (Section 4). We then ground the discussion in the health span concept (Section 5) and outline a research agenda (Section 6). Throughout, we draw on WHO implementation guidance, systematic evidence on care models for multimorbidity, and the equity literature on priority setting.

\subsection*{Ensemble Context}

\textbf{Paper ID:} \texttt{80fbe599-356c}

This paper is part of a four-paper ensemble on health evaluation under indefinite horizons. The root paper (\texttt{c2f3de53-f086}) defines the core mathematical framework---QALY flow as a renewal-reward ratio, with formal results on existence, cost-effectiveness, perturbation stability, and distributional aggregation. This paper translates that framework into public health practice: integrated care as the delivery backbone, maintenance burden as a policy variable, and equity as a structural concern under sustained access. Companion papers explore the civilizational context of longevity (\texttt{73997063-e1ed}) and provide cross-scale synthesis connecting aging biology, reliability engineering, and governance (\texttt{d7006442-c67e}).

\section{Integrated care as the delivery backbone}

\subsection{The ICOPE continuum}

The WHO Integrated Care for Older People (ICOPE) handbook operationalizes healthy ageing through structured workflows rooted in primary care \citep{WHO2024ICOPE}. The second edition (2024) provides step-by-step guidance for implementation across diverse resource settings. Its core elements form a continuous cycle that begins with rapid screening for declines in intrinsic capacity---mobility, cognition, sensory function, psychological wellbeing, and vitality---followed by comprehensive assessment when screening indicates concern. From assessment, the process moves to person-centred care planning: the development of individualized plans oriented toward functional goals that matter to the person, incorporating their preferences, values, and life context rather than imposing standardized disease-management protocols. These plans are then enacted through multidisciplinary care delivery that coordinates across medicine, nursing, rehabilitation, nutrition, pharmacy, social work, and community health workers, with clear role definitions so that no aspect of the person's needs falls through gaps between professions. The cycle continues with scheduled monitoring and follow-up, reassessing function and revising care plans as capacity changes, ensuring continuity across acute episodes and care transitions that might otherwise fragment the person's experience. Finally, when functional decline exceeds what community-based services can address, clear pathways link the person to more intensive long-term care supports, whether home-based or residential.

This architecture already embodies the flow framework's logic: value accrues through the ongoing process of maintaining function, and the system's job is to sustain that process reliably over time. ICOPE provides the delivery infrastructure that any long-horizon intervention---including future ageing-modifying therapies---would require. Novel treatments would be embedded within these workflows, adding new nodes (periodic infusions, genetic monitoring, regenerative procedures) to an existing continuum rather than replacing it.

\subsection{What the evidence on multimorbidity care reveals}

As people live longer, chronic multimorbidity becomes the norm rather than the exception. Managing health for individuals with multiple interacting conditions requires ongoing orchestration across providers, settings, and time. The evidence base on multimorbidity care models illuminates what sustained delivery demands.

A scoping review of systematic reviews on integrated care for multimorbidity finds wide variation in how models are defined, what components they include, and how outcomes are measured \citep{Rohwer2023IntegratedCareScoping}. Models differ in their target populations (disease-specific versus condition-agnostic), care settings (primary care, hospital, community), integration mechanisms (case management, shared records, multidisciplinary teams), and outcome domains (clinical endpoints, patient experience, resource use). This heterogeneity reflects genuine contextual variation, but it also reveals a gap: without standardized descriptions, we cannot reliably predict which models will sustain function over long horizons.

A complementary review focused on low- and middle-income settings emphasizes that outcomes depend heavily on organizational arrangements \citep{Endalamaw2024MultimorbidityCareModels}. Effective models require leadership and governance structures that establish clear accountability for the care continuum, with authority to coordinate across the silos that typically fragment health systems. They require workforce designs that define team roles clearly, train providers in chronic care competencies distinct from acute medicine, and maintain staffing ratios adequate for the follow-up visits that sustained care demands. Financing mechanisms must reward continuity and outcomes over time rather than episodic throughput; fee-for-service payment that incentivizes volume undermines the steady, unglamorous work of maintenance. Information systems---shared electronic records, population registries, automated reminders for overdue follow-up, decision support for care transitions---provide the connective tissue that prevents patients from falling through gaps between providers or being lost when they move. Finally, effective models require systematic patient identification and recruitment: actively seeking out individuals who would benefit from integrated care rather than passively waiting for them to present with complications.

These organizational elements determine whether a care model can sustain benefit delivery. An intervention with strong efficacy in controlled conditions may fail in practice if the health system cannot maintain follow-up, coordinate transitions, or retain trained staff. The flow framework captures this directly: QALY flow depends on system sustainability, and system sustainability depends on organizational infrastructure.

\subsection{Implications for long-horizon planning}

Flow-based evaluation reorients planning around delivery sustainability. The key questions shift:

\begin{table}[h]
\centering
\caption{Evaluation focus under flow-based thinking}
\begin{tabular}{@{}p{3.5cm}p{5cm}p{5cm}@{}}
\toprule
\textbf{Dimension} & \textbf{Episodic framing} & \textbf{Flow-based framing} \\
\midrule
Primary outcome & Lifetime QALYs gained & QALY flow rate sustained \\
Time horizon & Fixed (e.g., 10-year model) & Indefinite regime \\
Cost metric & Total expenditure & Cost flow (expenditure per person-time) \\
Success criterion & Positive incremental net benefit & Higher sustained flow at acceptable cost flow \\
System capacity & Implicit background assumption & Explicit constraint and outcome \\
Failure modes & Treatment non-response & Discontinuity, dropout, system strain, coverage gaps \\
\bottomrule
\end{tabular}
\end{table}

This reorientation has practical consequences. Interventions with low maintenance burden become more attractive. Delivery models with strong follow-up mechanisms score higher. Data requirements shift toward persistence, adherence, and long-term safety rather than short-term efficacy alone. Planning must account for workforce pipelines, supply chain resilience, and financing stability over decades.

The ICOPE framework and the multimorbidity evidence base provide the foundation. Extending horizons strengthens and scales the delivery architectures already in development.

\section{Maintenance burden as a policy variable}

\subsection{What is continuous with current practice}

Public health already thinks in maintenance terms. Vaccination programs require boosters. Chronic disease management involves ongoing monitoring and medication adjustment. Nutrition and wellness programs aim to sustain healthy behaviours over years. Screening programs repeat at intervals. The concept of maintenance burden is native to the field.

What extended horizons change is the quantitative weight of maintenance considerations, not their qualitative character. Three differences matter:

\paragraph{Duration amplification.} A recurring cost that consumes 1\% of a person's health budget annually represents 10\% of lifetime resources over ten years but 50\% over fifty years. Burdens that are tolerable over current planning horizons can become dominant as horizons extend. The same intervention that appears cost-effective in a ten-year model may appear unsustainable in a fifty-year model.

\paragraph{Cumulative risk.} A low-probability adverse event that occurs with 0.5\% annual probability produces a 5\% cumulative risk over ten years but a 22\% risk over fifty years. Events that are rare in current time frames become likely or near-certain as horizons extend. Pharmacovigilance, redundancy, and fail-safe design become more consequential.

\paragraph{System scaling.} Current wellness and chronic care infrastructure is sized for populations that require support for roughly forty years of adult life. If healthy adulthood extends to eighty or one hundred years, the same per-person-year burden produces twice or more the aggregate system demand. Workforce pipelines, training capacity, and capital investment must scale accordingly.

The contribution of the flow framework is to make these duration effects explicit in evaluation. Cost flow (expenditure per person-time) and QALY flow (health per person-time) remain stable metrics as horizons vary, allowing comparisons that do not depend on assumptions about when life ends.

\subsection{Components of maintenance burden}

Every intervention that requires repetition imposes ongoing demands on patients, caregivers, health workers, and systems. These demands constitute maintenance burden, and they enter the flow framework directly as cost flow: the resource expenditure per unit person-time required to sustain a given level of QALY flow.

Maintenance burden spans multiple domains:

\paragraph{Clinical encounters.} Visits for monitoring, re-dosing, imaging, laboratory work, and specialist consultation. A therapy requiring quarterly infusions imposes four clinical encounters per year; over a thirty-year maintenance horizon, that amounts to 120 infusion visits per patient, each consuming clinical time, supplies, and scheduling capacity.

\paragraph{Adherence requirements.} Medication regimens, dietary protocols, exercise programs, behavioural commitments. Complex regimens with multiple daily doses, timing constraints, or dietary restrictions increase the cognitive and logistical load on patients. Adherence declines predict treatment failure; supporting adherence requires counselling, reminders, and sometimes technology.

\paragraph{Supply chains.} Manufacturing capacity, distribution networks, cold chains for biologics, consumables. Supply disruptions can interrupt treatment for entire populations. Rare or expensive inputs create vulnerability. Generic availability and local production capacity affect long-run sustainability.

\paragraph{Workforce.} Specialists to prescribe, generalists to monitor, nurses to administer, pharmacists to dispense, care coordinators to follow up, community health workers to bridge gaps. Workforce training pipelines, retention, and geographic distribution constrain what can be delivered at scale.

\paragraph{Information systems.} Electronic health records, registries for tracking eligible populations, scheduling systems, alert mechanisms for overdue follow-up, decision support for care transitions. Robust information infrastructure prevents patients from being lost to follow-up and supports coordination.

\paragraph{Patient and caregiver time.} Travel to appointments, waiting, recovery from procedures, administrative tasks. For conditions requiring ongoing management, these time costs accumulate. They fall disproportionately on those with fewer resources, less flexible employment, and weaker social support.

\subsection{Compounding over extended horizons}

Small recurring burdens compound. An intervention requiring one additional specialist visit per year appears modest over a five-year trial horizon. Over a fifty-year maintenance horizon, it represents fifty visits for each patient---and the associated infrastructure, workforce, and financing demands multiplied across the covered population.

The flow framework surfaces this compounding automatically. High maintenance burden appears as high cost flow. Interventions with low efficacy relative to their maintenance demands produce poor cost-effectiveness ratios when evaluated as flow, even if they appeared acceptable under finite-horizon summation.

This has implications for technology assessment. Interventions should be evaluated in terms of the sustained rate of benefit they produce (QALY flow) relative to the sustained rate of resources they consume (cost flow). The ratio is the flow-based incremental cost-effectiveness ratio: ongoing cost per unit of ongoing benefit. This metric remains stable as horizons extend, enabling comparison across interventions with different durability profiles.

\subsection{Catastrophic risk and truncation}

Long horizons magnify the importance of low-probability, high-severity events. Adverse events from repeated treatment---cumulative toxicity, rare but severe immunological reactions, organ damage that emerges only after years of exposure---become more likely as treatment duration extends. Discontinuities in coverage interrupt the maintenance stream: insurance gaps when employment changes, loss of eligibility when crossing jurisdictional boundaries, policy reversals when governments change. Supply disruptions can halt treatment for entire populations, whether from manufacturing failures, geopolitical shocks that sever trade routes, or pandemic-related shortages that redirect production capacity. Coordination failures---missed handoffs when patients transition between care settings, fragmented records that leave new providers ignorant of treatment history, turnover among the personnel who know a patient's situation---create gaps through which patients fall. And abrupt transitions to severe disability, from stroke, major injury, or rapid cognitive decline, can end the maintenance relationship entirely by destroying the functional capacity that treatment aimed to preserve.

In the flow framework, catastrophic events truncate the cycle: when catastrophe occurs, the individual's contribution to health production ends. For the metric to behave well, policies must attend to catastrophe rates. Reducing the probability of truncation increases expected cycle length, which---for a given quality level---increases QALY flow.

Catastrophe prevention thus becomes a first-order policy objective. This includes pharmacovigilance for long-term treatment effects, redundancy in supply chains, portability of coverage across jurisdictions, robust handoff protocols, and prevention of disabling events (falls, cardiovascular events, cognitive decline). These are core determinants of long-run health production.

\section{Equity in sustained access}

\subsection{How inequity compounds---and how catch-up becomes possible}

When benefits accrue episodically, the maximum equity gap is the difference in outcomes from one episode. When benefits accrue as a continuous stream, inequity compounds. Those with sustained access accumulate healthy person-time over years and decades; those without access fall progressively further behind.

Current disparities in chronic disease management already demonstrate compounding. Populations with poor access to primary care accumulate preventable complications: uncontrolled hypertension leads to stroke, unmanaged diabetes leads to nephropathy, missed cancer screenings lead to late-stage diagnosis. Each missed opportunity for intervention narrows future options. Extended horizons amplify these dynamics: a twenty-year gap in access to an ageing-modifying therapy could produce a difference of twenty healthy years.

Extended horizons also create space for recovery. Under a fixed lifespan, someone who falls behind has limited remaining time to catch up. Under an extended horizon, late access to effective interventions leaves more future time during which benefits can accumulate. A person who gains access to maintenance therapy at age sixty rather than forty still has decades of potential benefit ahead if the horizon extends to one hundred or beyond. This cuts both ways for policy: compounding makes early access more valuable, but extended horizons make late interventions more worthwhile than they would be under fixed-lifespan assumptions. The equity imperative is to prevent permanent exclusion, because exclusion forecloses the catch-up that extended horizons otherwise make possible.

\subsection{Priority setting under extended horizons}

Priority setting frameworks that incorporate equity become more salient when horizons extend. Traditional cost-effectiveness analysis treats a QALY as a QALY regardless of who receives it. Equity-weighted approaches assign greater value to health gains for worse-off individuals or groups.

The proportional shortfall approach provides one vocabulary for these debates \citep{Mahdiani2024ProportionalShortfall}. Severity is measured as the proportion of expected healthy lifetime that a person will lose due to their condition. Interventions that address greater proportional shortfall receive higher priority weight. This approach has been adopted in several national priority-setting contexts and connects to broader debates about fairness in health resource allocation.

Under extended horizons, proportional shortfall gains new dimensions. If healthy life expectancy itself becomes a function of access to maintenance therapies, then shortfall calculations must account for the treatment landscape. Someone denied access to an effective ageing-modifying intervention would experience a larger shortfall than someone with access. Equity-weighted evaluation would assign greater priority to ensuring access for excluded groups.

The flow framework supports explicit equity weights. Define QALY flow for each population subgroup. Apply distributional weights that reflect social priority for improvements in worse-off groups. Aggregate weighted flows. Evaluate policies by their impact on the weighted aggregate and on the distribution (e.g., the minimum flow across groups).

\subsection{Structural commitments for equity}

Equity in flow-based evaluation requires structural commitments that go beyond individual coverage decisions:

\paragraph{Floor guarantees.} Establish minimum QALY flow levels below which no population subgroup should fall. This operationalizes a commitment to ensuring that everyone reaches a threshold of sustained health, regardless of ability to pay or geographic location.

\paragraph{Access universality.} Design maintenance systems for population coverage rather than market segmentation. If an intervention requires ongoing delivery, access must be continuous. Gaps in coverage interrupt treatment and undermine health production.

\paragraph{Continuous eligibility.} Enrollment structures should avoid periodic requalification requirements that create administrative barriers and dropout. Once someone is on a maintenance regimen, continuity should be the default.

\paragraph{Explicit priority weights.} Cost-effectiveness analysis should incorporate explicit equity weights. The Netherlands, Norway, and other jurisdictions have implemented severity-weighted priority setting \citep{Mahdiani2024ProportionalShortfall}. Extended horizons make these weights more consequential and require updating them to reflect the new stakes.

\paragraph{Governance for the long run.} Commitments must be durable. Policies that promise maintenance access but are vulnerable to budget cycles, political transitions, or institutional change will fail to deliver sustained benefit. Governance structures---constitutional protections, independent health authorities, long-term funding mechanisms---matter for equity.

\section{Health span as the grounded endpoint}

\subsection{From life span to health span}

Public health audiences reasonably resist claims of radical life extension. The history of longevity science includes cycles of hype and disappointment. A constructive alternative is to focus on health span: the period of life spent in good functional health.

The health span concept has gained traction in mainstream gerontology. Olshansky and colleagues argue that major gains in life expectancy have already been achieved through public health and medicine, and that the priority now should be extending the healthy portion of life rather than total survival \citep{Olshansky2022HealthSpan}. Compressing morbidity into a shorter period at the end of life would improve quality of life and reduce the burden of disability and dependency.

This framing aligns naturally with the UN Decade's emphasis on functional ability \citep{WHO2020DecadePlan}. Success is measured by what people can do, by their capacity to engage in valued activities, by their participation in community life. Biological markers of ageing are relevant only insofar as they predict functional outcomes.

\subsection{QALY flow as health span production}

QALY flow, when quality is interpreted as functional capacity, measures exactly what the health span concept targets: the sustained rate at which a population produces healthy, functional person-time. High flow means that, on average, individuals spend their time in high-function states. The policy objective is to maximize this rate.

This interpretation has several advantages:

\paragraph{Connects longevity science to public health.} Framing the goal as health span production makes clear that longevity research and public health share a common objective. Both aim to extend the period of functional health. The difference is in the tools (biomedical versus environmental, clinical versus population-level).

\paragraph{Avoids hype.} Targeting health span requires only the plausible claim that the ratio of healthy to unhealthy years can be improved. This is already happening: age-specific disability rates have declined in many countries, and active life expectancy has increased \citep{Olshansky2022HealthSpan}.

\paragraph{Centers lived experience.} Functional ability is about what people can do: move, see, hear, think, remember, engage. This is the substance of quality of life. Biomarkers are proxies; function is the outcome that matters.

\paragraph{Connects to disability and rehabilitation.} The health span frame naturally incorporates disability prevention, rehabilitation, assistive technology, and supportive environments. These are core public health and health systems functions. Extending health span means reducing the incidence of disabling conditions, improving recovery when they occur, and providing support that maintains function despite impairments.

\subsection{Reframing ageing-modifying interventions}

Future ageing-modifying therapies---senolytic drugs, NAD+ precursors, cellular reprogramming, gene therapies---can be evaluated within this frame. The question is: Does this increase QALY flow? An intervention that extends survival while maintaining or improving average quality increases flow. An intervention that extends survival in low-quality states may not.

This provides a principled basis for evaluation. It also connects novel therapies to existing health technology assessment methods. Cost-effectiveness analysis using QALY flow ratios can compare an ageing-modifying therapy to conventional preventive interventions, chronic disease treatments, or care model improvements. All are evaluated by the same metric: sustained health per unit of resources.

\section{A research agenda}

\subsection{Measurement and surveillance}

Flow-based planning requires indicators that track durable functional ability at the population level. The UN Decade framework identifies domains for measurement: intrinsic capacity (physical, cognitive, sensory, psychological), functional ability (mobility, self-care, communication, learning, social participation), and environments that support ability \citep{WHO2020DecadePlan}.

Current surveillance systems emphasize point-in-time prevalence: the proportion of the population with a condition at a given moment. Flow-based thinking adds emphasis on rate measures: how much functional person-time is the population producing per unit of calendar time? This requires longitudinal data on health states, transitions, and durations.

System indicators are equally important. Can the health system sustain delivery? Key metrics include continuity of care (proportion of patients with a regular source of care, follow-up rates), access (wait times, geographic coverage, financial barriers), workforce capacity (provider ratios, vacancy rates, burnout), and supply chain resilience (stockout rates, supplier diversity).

\subsection{Evaluation design}

Flow-based evaluation requires evidence on persistence and sustainability. Current trial designs typically assess efficacy at fixed endpoints---one year, three years---then stop following patients. Long-horizon planning requires extending that evidence base in several directions. Long-term follow-up extensions should continue trials beyond the primary endpoint to observe whether benefits persist, wane, or reverse, and to detect safety signals that emerge only after years of exposure. Pragmatic evaluations should embed trials in routine practice settings to assess effectiveness under real-world delivery conditions, where adherence is imperfect, patients drop out, and system constraints limit what clinicians can do. These complement explanatory trials that establish efficacy under ideal conditions. Real-world evidence from registries, claims data, and electronic health records can track outcomes over years and decades at population scale, with attention to the discontinuations, treatment switching, and losses to follow-up that trials often exclude. Maintenance burden measurement should become standard: systematically quantifying all resource categories---clinical encounters, patient time, supply chain requirements, workforce demands---required to sustain treatment, not just the drug cost that dominates current economic evaluation. And failure mode analysis should study what actually causes treatment discontinuation, coverage gaps, and adverse outcomes over extended periods, treating system breakdowns as objects of inquiry rather than noise to be excluded from analysis.

\subsection{Implementation science for integrated care}

The evidence base on integrated care for multimorbidity reveals wide variation in model components, contexts, and outcomes \citep{Rohwer2023IntegratedCareScoping}. This variation must be reduced to enable planning.

Several research priorities follow. Standardized model description is foundational: developing taxonomies and reporting standards that specify what an integrated care model actually includes---its target population, care setting, team composition, coordination mechanisms, information systems, and financing arrangements---so that studies become comparable and findings become transferable. Context documentation should accompany every evaluation, systematically describing the implementation environment: health system structure, resource levels, workforce availability, and policy environment. Without context, we cannot know whether a model's success or failure reflects the intervention itself or the conditions under which it was tried. Scalability and sustainability research should focus specifically on what happens when models are scaled beyond pilot sites and maintained over years, since the dynamics of a well-resourced demonstration project differ fundamentally from those of routine system-wide operation. Adaptation research should examine how models developed in high-income settings can be modified for low- and middle-income contexts, where the majority of the world's ageing population lives and where health system constraints are most binding \citep{Endalamaw2024MultimorbidityCareModels}.

The ICOPE framework provides a starting point for standardization \citep{WHO2024ICOPE}. Implementation research can build on this foundation to identify which elements are essential, which are adaptable, and which contextual factors determine success.

\subsection{Governance and legitimacy}

Extended-horizon interventions raise questions that technical evaluation cannot answer. Intergenerational fairness becomes more fraught when benefits accrue over decades: who pays now for health gains that will materialize in thirty or fifty years? How should the costs and benefits of maintenance regimes be distributed across age cohorts, when the young may be asked to fund therapies whose full value they will not experience until they themselves are old? Commitment credibility is equally challenging: will governments, insurers, and health systems honour long-term access commitments, or will future budget pressures, political transitions, and institutional changes erode promises made today? What institutional arrangements---constitutional provisions, independent health authorities, dedicated funding streams---can make commitments durable across the political cycles that typically govern health policy? Eligibility boundaries must be drawn and redrawn: who qualifies for maintenance therapies, and how are those decisions revisited as technology evolves and new interventions become available? Public trust underlies all of this: how do populations come to believe that long-term commitments will actually be kept, and what transparency and accountability mechanisms can sustain that trust across the decades over which maintenance relationships must endure?

These questions require deliberative processes: citizen panels, ethics committees, legislative debate, and international coordination. Public health institutions should engage these questions early, before policy becomes reactive to hype or backlash.

\section{Conclusion}

A future in which ageing becomes more manageable deepens the relevance of public health. The central challenge shifts from curing conditions to sustaining functional ability fairly over long horizons. Meeting that challenge requires integrated care systems designed for continuous delivery, systems whose architecture the ICOPE framework and multimorbidity care evidence already provide \citep{WHO2024ICOPE, Endalamaw2024MultimorbidityCareModels}. It requires explicit accounting for maintenance burden, recognizing that every repeating intervention imposes ongoing demands on patients, providers, and systems, and that flow-based evaluation surfaces these costs rather than obscuring them behind one-time expenditure figures. It requires equity structures that prevent compounding disadvantage: floor guarantees below which no population should fall, universal access to maintenance regimes, continuous eligibility that does not force periodic requalification, and explicit priority weights that direct resources toward those falling behind \citep{Mahdiani2024ProportionalShortfall}. It requires research that tracks persistence and sustainability---long-term follow-up extensions, pragmatic trials in routine settings, real-world evidence from registries and records, and systematic analysis of failure modes. And it requires governance capable of making and keeping long-term commitments: deliberative processes that build legitimacy, durable institutions that outlast political cycles, and the public trust that comes from transparency and accountability.

The QALY flow framework provides the evaluative language for this transition \citep{Framework2026}. The UN Decade of Healthy Ageing provides the policy platform \citep{WHO2020DecadePlan}. The health span concept provides the grounded endpoint \citep{Olshansky2022HealthSpan}. The work ahead is to build the systems, generate the evidence, and establish the governance structures that can deliver sustained health over extended horizons.

\newpage
\bibliographystyle{plainnat}
\bibliography{references}

\end{document}
