\documentclass[11pt]{article}

\usepackage[margin=1in]{geometry}
\usepackage{amsmath, amssymb, amsthm, mathtools}
\usepackage{bbm}
\usepackage{hyperref}
\usepackage{orcidlink}
\usepackage{enumitem}
\usepackage{natbib}

\hypersetup{
  colorlinks=true,
  linkcolor=blue,
  citecolor=blue,
  urlcolor=blue
}

% --- Notation ---
\newcommand{\R}{\mathbb{R}}
\newcommand{\E}{\mathbb{E}}
\newcommand{\Pp}{\mathbb{P}}
\newcommand{\1}{\mathbbm{1}}
\newcommand{\cS}{\mathcal{S}}
\newcommand{\cA}{\mathcal{A}}
\newcommand{\cX}{\mathcal{X}}
\newcommand{\cI}{\mathcal{I}}
\newcommand{\cR}{\mathcal{I}}
\newcommand{\absorb}{\dagger}
\newcommand{\TV}{\mathrm{TV}}

% --- Theorem environments ---
\newtheorem{axiom}{Axiom}
\newtheorem{assumption}{Assumption}
\newtheorem{definition}{Definition}
\newtheorem{theorem}{Theorem}
\newtheorem{proposition}{Proposition}
\newtheorem{corollary}{Corollary}
\newtheorem{lemma}{Lemma}
\theoremstyle{remark}
\newtheorem{remark}{Remark}

\title{Evaluating Health Policy Without a Fixed Lifespan: A Steady-State Foundation for Quality, Maintenance, and Risk}
\author{Daniel S. Goldman \orcidlink{0000-0003-2835-3521}}
\date{\today}

\begin{document}
\maketitle

\begin{abstract}
In a world where biological lifespan has no fixed upper bound, yet individuals remain vulnerable to stochastic catastrophic failures (accidents, disease, irreparable breakdown) and also experience repairable degradation requiring ongoing maintenance, lifetime-summed health metrics become structurally ill-suited for public policy. This note gives a foundation for evaluation based on \emph{steady per-unit-time} health value: a regenerative (renewal) QALY flow defined as a ratio of expected cycle reward to expected cycle duration, together with compatible notions of cost-effectiveness, stability under perturbations, and explicit extensions for irreversibility and risk/volatility. The framework is stated as axioms plus a controlled continuous-time model and yields standard existence and sensitivity results under transparent conditions.
\end{abstract}

\tableofcontents

\section{Setting and design requirements}

Health technology assessment relies on established frameworks for comparing interventions \citep{sanders2016_secondpanel}, but these presuppose finite planning horizons. As longevity science advances the prospect of radical healthspan extension \citep{masny2023_healthspanjustice}, fundamental questions arise about how to evaluate policies when lifespans may be unbounded. The extensive literature on discounting in health economics \citep{attema2018_discounting, gravelle2007_discounting} has debated time preference, but unbounded horizons require a more fundamental rethinking.

\subsection{The setting}
We assume the following stylized world:
\begin{itemize}[leftmargin=2em]
\item There is \emph{no fixed biological maximum lifespan}. Conditional on avoiding catastrophic failure and receiving ongoing care, individuals can persist indefinitely.
\item Catastrophic failures still occur stochastically (accidents, lethal disease, irreparable system failure).
\item Many degradations are \emph{repairable}, with maintenance and replacement restoring function at monetary, time, and quality costs.
\end{itemize}

Public policy must allocate scarce resources among prevention, monitoring, maintenance, repair technologies, environments, and protections against catastrophic risks.

\subsection{Axioms (explicitly normative and structural)}
We state the requirements as axioms. Some are structural (needed for a usable metric), others encode transparent normative choices. Axioms~\ref{ax:horizon}--\ref{ax:death} in particular reflect welfarist and experientialist commitments that, while standard in health economics, are substantive ethical positions rather than purely structural requirements.

\begin{axiom}[Well-posedness]\label{ax:wellposed}
Policy evaluation must be finite for all admissible policies and must support stable marginal comparisons (e.g.\ cost per unit health) under arbitrarily long horizons.
\end{axiom}

\begin{axiom}[Horizon invariance for ongoing regimes]\label{ax:horizon}
The criterion should evaluate an ongoing health regime rather than depend on an exogenously fixed terminal time.
\end{axiom}

\begin{axiom}[Experienced welfare as a flow]\label{ax:flow}
Health value accrues through an instantaneous quality-of-life quantity $q(t)$ (possibly reduced during downtime/repair), so the experienced path of health states matters.
\end{axiom}

\begin{remark}
Axiom~\ref{ax:flow} reflects standard practice in health technology assessment, where QALYs serve as the primary outcome measure \citep{sanders2016_secondpanel, brazierrowen2011_tsd11}.
\end{remark}

\begin{axiom}[Death as truncation of future experience]\label{ax:death}
Catastrophic failure matters because it ends future experienced welfare; the criterion should not introduce independent terminal utilities that dominate the analysis.
\end{axiom}

\begin{axiom}[Explicit normativity]\label{ax:explicit}
Distributional priorities, inequality aversion, unacceptable-quality thresholds, and any time preference (if used) must appear as explicit parameters rather than being hidden inside technical devices.
\end{axiom}

\begin{remark}
This axiom contrasts with discounted approaches where time preference is embedded in the discount rate \citep{attema2018_discounting, gravelle2007_discounting}. Distributional cost-effectiveness analysis \citep{asaria2013_dcea} makes equity weights explicit; we adopt the same transparency principle throughout.
\end{remark}

\begin{axiom}[Policy-relevance as cross-sectional person-time]\label{ax:persontime}
Public-health objectives should be interpretable as welfare per unit of population time (person-time) under a stable institutional regime.
\end{axiom}

\begin{remark}
Axiom~\ref{ax:persontime} is the key bridge from individual life histories to public policy: governments repeatedly face a stream of individuals and person-time exposed to hazards and maintenance systems. A per-person-time objective matches what policy instruments actually control.
\end{remark}

\begin{remark}[On interpersonal aggregation]
The person-time interpretation implies fungibility of welfare across individuals, treating one person's quality-time as substitutable for another's. This is not a novel commitment of the present framework but is inherited from standard QALY-based health economics, which already aggregates health gains across persons. In a world of unbounded lifespans, this assumption becomes \emph{more} defensible: as individual life trajectories extend indefinitely, the probability that any two individuals will experience similar health states grows, and interpersonal comparisons become less problematic due to trajectory convergence over long horizons.
\end{remark}

\section{A controlled health process with repairability and catastrophe}

\subsection{State space, quality, and irreversibility}
Let $\cS$ denote the set of \emph{alive} health states. Let $\absorb$ denote catastrophic failure (death/irreparable loss), which is absorbing. Define $\cX \coloneqq \cS \cup \{\absorb\}$. This multi-state formulation follows the health economics tradition of modeling health trajectories as stochastic processes \citep{bauer_lakdawalla_reif2025_healthrisk, song_bauer_lakdawalla_reif2023_value}.

A measurable function $q:\cX\to[0,1]$ gives instantaneous quality, with $q(\absorb)=0$. The quality function assigns health-related quality of life to each state, following established measurement frameworks \citep{brazierrowen2011_tsd11}.

We also designate a subset $\cI\subseteq \cS$ of \emph{irreversible harm} states and $\cR \coloneqq \cS\setminus \cI$ of \emph{reversible/repairable} states. This partition is conceptual and policy-relevant: interventions can reduce transitions into $\cI$, increase repair success out of $\cR$, or alter time spent in low-quality downtime states.

\subsection{Actions and costs}
For each $x\in\cS$, let $\cA(x)$ be a nonempty set of admissible actions (prevention intensity, monitoring, repair protocol, environment, replacement choice). A (possibly history-dependent) policy $\pi$ selects actions over time.

Let $c:\cS\times \cA \to [0,\infty)$ be a measurable running cost; set $c(\absorb,\cdot)=0$.

\subsection{Dynamics}
We model health dynamics as a controlled continuous-time process $(X_t)_{t\ge 0}$ on $\cX$ (e.g.\ a controlled CTMC or more general controlled jump process). Under action $a\in \cA(x)$ at alive state $x$, transitions are governed by a rate kernel $Q(\cdot\mid x,a)$ with $\absorb$ absorbing:
\[
Q(\absorb\mid \absorb,a)=0,\qquad Q(y\mid \absorb,a)=0\;\;\forall y\neq \absorb.
\]

Define the catastrophic time
\[
\tau \coloneqq \inf\{t\ge 0: X_t=\absorb\}.
\]

\begin{assumption}[Bounded quality and well-defined dynamics]\label{as:basic}
$q(x)\in[0,1]$ for all $x\in\cX$ with $q(\absorb)=0$. Under policies of interest, the controlled process is non-explosive (finitely many jumps on finite time intervals almost surely).
\end{assumption}

\begin{remark}[Sufficient conditions for non-explosiveness]
Non-explosiveness holds if transition rates are uniformly bounded: $\sup_{x\in\cS,a\in\cA(x)} Q(\cX\setminus\{x\}\mid x,a) < \infty$. More generally, for countable $\cS$, standard Foster--Lyapunov criteria apply: if there exists a function $V:\cS\to[1,\infty)$ with $V(x)\to\infty$ as $x\to\infty$ (in some ordering) and constants $c,d>0$ such that $\sum_{y} Q(y\mid x,a)(V(y)-V(x)) \le c - d\cdot Q(\cX\setminus\{x\}\mid x,a)$ for all $x,a$, then the process is non-explosive.
\end{remark}

\section{Regeneration and the steady QALY flow objective}

\subsection{The regenerative regime axiom (institutional replacement)}
To interpret welfare per unit person-time, we work with a regenerative regime: when a catastrophic loss occurs, the regime continues with a new entrant. This matches population-level policy evaluation (birth, immigration, institutional replacement, workforce turnover).

\begin{assumption}[Regeneration / replacement]\label{as:regen}
Fix an initial distribution $\nu$ on $\cS$. Each time a catastrophic event occurs, a new individual enters immediately with initial state distributed as $\nu$, independent of the past, and is governed by the same policy $\pi$.
\end{assumption}

\begin{remark}
Assumption~\ref{as:regen} is the formal expression of Axiom~\ref{ax:persontime}. It does not claim that any particular individual is replaced; it defines the unit of analysis for public policy: a continuing social process generating person-time under an ongoing regime.
\end{remark}

\begin{remark}[Large-population interpretation]
The regeneration assumption is best understood as modeling a large stationary population where catastrophic failures and new entrants are rare and approximately uncorrelated. In this interpretation:
\begin{itemize}[leftmargin=2em]
\item The regenerative cycle represents the expected trajectory of a randomly sampled individual from the cross-section.
\item The ratio $J(\pi)$ equals the cross-sectional average quality at any point in time under demographic steady-state.
\item ``Immediate replacement'' reflects the continuous demographic flow in a large population, not a claim about literal individual replacement.
\item ``Independence'' follows from population size: any single death is statistically uncorrelated with the vast pool of existing individuals.
\item ``Identical initial distribution'' reflects the stationary assumption---we are modeling the steady-state regime, not the transition from bounded to unbounded lifespans.
\end{itemize}
This interpretation dissolves concerns about the literal plausibility of i.i.d.\ replacement by reframing the assumption as a consequence of the modeling target (cross-sectional welfare in a stationary population) rather than an empirical claim about individual-level dynamics.
\end{remark}

\subsection{Cycle reward and cycle cost}
Under a policy $\pi$ and initial distribution $\nu$, define per-cycle quantities:
\begin{align}
R^\pi &\coloneqq \int_0^{\tau} q(X_t)\,dt, \label{eq:Rdef}\\
C^\pi &\coloneqq \int_0^{\tau} c(X_t,A_t)\,dt, \label{eq:Cdef}\\
\tau^\pi &\coloneqq \tau. \label{eq:taudef}
\end{align}
By Assumption~\ref{as:basic}, $0\le R^\pi \le \tau^\pi$ almost surely.

\subsection{Definition: QALY flow and cost flow}
The regenerative structure allows us to define a time-averaged objective that remains bounded and well-defined over infinite horizons. This addresses the fundamental challenge of evaluating policies when time horizons are unbounded \citep{omahony2015_time}.

\begin{definition}[Steady QALY flow and steady cost flow]\label{def:flow}
Assume $\E_\nu[\tau^\pi]\in(0,\infty)$. Define
\begin{equation}\label{eq:Jdef}
J(\pi) \;\coloneqq\; \frac{\E_\nu[R^\pi]}{\E_\nu[\tau^\pi]} \in [0,1],
\end{equation}
and
\begin{equation}\label{eq:Kdef}
K(\pi) \;\coloneqq\; \frac{\E_\nu[C^\pi]}{\E_\nu[\tau^\pi]} \in [0,\infty).
\end{equation}
\end{definition}

\begin{remark}[Risk neutrality in the base framework]
The objective $J(\pi)$ depends only on expected cycle reward and expected cycle duration, not on their variances or higher moments. This embeds a \emph{risk-neutral} preference: two policies with identical $J(\pi)$ but different cycle-to-cycle volatility in quality are treated as equally desirable. 

When risk aversion is desired, this can be addressed via the extensions in Section~\ref{sec:risk}: either by adding an explicit risk penalty (e.g., $J(\pi) - \lambda\sqrt{\text{Var}(\bar{q}^\pi)}$) or by applying coherent risk measures (e.g., CVaR) to cycle-level outcomes. The base framework is risk-neutral by design, leaving risk preferences as an explicit parameter to be chosen by the policy maker.
\end{remark}

\begin{proposition}[Basic boundedness]\label{prop:bounded}
If $\E_\nu[\tau^\pi]\in(0,\infty)$ then $J(\pi)\in[0,1]$.
\end{proposition}
\begin{proof}
From $0\le R^\pi \le \tau^\pi$ a.s., taking expectations yields $0\le \E[R^\pi]\le \E[\tau^\pi]$, hence $J(\pi)\in[0,1]$.
\end{proof}

\subsection{Theorem: long-run person-time interpretation}
Under Assumption~\ref{as:regen}, successive cycles are i.i.d.\ with cycle lengths $\tau_1,\tau_2,\dots$ and cycle rewards $R_1,R_2,\dots$ distributed as $(\tau^\pi,R^\pi)$.

Let $Q(T)\coloneqq \int_0^T q(X_t)\,dt$ be cumulative quality in the regenerative process.

\begin{theorem}[Renewal-reward representation]\label{thm:renewal}
Assume $\E_\nu[\tau^\pi]\in(0,\infty)$. Then
\[
\frac{Q(T)}{T}\xrightarrow[T\to\infty]{a.s.} J(\pi),
\qquad\text{and}\qquad
\E\!\left[\frac{Q(T)}{T}\right]\to J(\pi).
\]
\end{theorem}

\begin{remark}
Theorem~\ref{thm:renewal} is the central justification: $J(\pi)$ is the asymptotic average quality per unit person-time sustained by policy $\pi$ under a continuing regime. This directly satisfies Axioms~\ref{ax:wellposed}--\ref{ax:persontime}.
\end{remark}

\subsection{Handling regimes with very long or infinite expected cycles}
Unbounded lifespan allows policies for which $\E_\nu[\tau^\pi]=\infty$. Rather than impose a discount rate to force finiteness, we adopt an explicit truncation-based normalization that preserves Axiom~\ref{ax:explicit}.

\begin{definition}[Truncated steady QALY flow]\label{def:trunc}
For $T>0$, define
\[
R_T^\pi \coloneqq \int_0^{\tau\wedge T} q(X_t)\,dt,\qquad
\tau_T^\pi \coloneqq \tau\wedge T,
\]
and define
\[
J^+(\pi)\coloneqq \limsup_{T\to\infty}\frac{\E_\nu[R_T^\pi]}{\E_\nu[\tau_T^\pi]},
\qquad
J^-(\pi)\coloneqq \liminf_{T\to\infty}\frac{\E_\nu[R_T^\pi]}{\E_\nu[\tau_T^\pi]}.
\]
When $J^+(\pi)=J^-(\pi)$, denote the common value by $J(\pi)$.
\end{definition}

\begin{proposition}[Well-posedness of truncated flow]\label{prop:truncwellposed}
For all $T>0$ and all policies $\pi$, $\E_\nu[\tau_T^\pi]>0$.
\end{proposition}
\begin{proof}
Since $\tau_T^\pi = \tau\wedge T \ge \min(\tau, T)$ and $\Pp(\tau > 0) = 1$ (the process starts in an alive state), we have $\tau_T^\pi > 0$ a.s.\ for $T>0$, hence $\E[\tau_T^\pi] > 0$.
\end{proof}

\begin{proposition}[Truncated flow is always bounded]\label{prop:truncbounded}
For all $\pi$, $J^+(\pi),J^-(\pi)\in[0,1]$.
\end{proposition}
\begin{proof}
As before, $0\le R_T^\pi \le \tau_T^\pi$ a.s., hence $0\le \E[R_T^\pi]/\E[\tau_T^\pi]\le 1$ for each $T$, so the $\limsup$ and $\liminf$ lie in $[0,1]$.
\end{proof}

\begin{proposition}[Convergence of truncated flow]\label{prop:truncconv}
For stationary Markov policies on finite state spaces with $\absorb$ reachable from every state (i.e., for each $x\in\cS$ and action $a\in\cA(x)$, there exists a path to $\absorb$ with positive probability), $J^+(\pi) = J^-(\pi)$, and this common value equals $J(\pi)$ as defined in \eqref{eq:Jdef}.
\end{proposition}
\begin{proof}
Under the reachability condition, $\Pp(\tau < \infty) = 1$ and $\E[\tau^\pi] < \infty$ (the expected hitting time of $\absorb$ is finite for finite irreducible chains with an absorbing state). By monotone convergence, $\E[R_T^\pi] \to \E[R^\pi]$ and $\E[\tau_T^\pi] \to \E[\tau^\pi]$ as $T\to\infty$. Hence the ratio converges: $\E[R_T^\pi]/\E[\tau_T^\pi] \to \E[R^\pi]/\E[\tau^\pi] = J(\pi)$.
\end{proof}

\section{Optimization under stationary policies}

\subsection{Stationary Markov policies}
Restrict attention to stationary Markov policies $f$ choosing $a=f(x)\in\cA(x)$ when in state $x$. Under such $f$, the process becomes time-homogeneous.

A policy objective can be scalarized as maximizing net steady welfare
\[
W_\lambda(f)\coloneqq J(f)-\lambda K(f),
\]
with $\lambda>0$ as an explicit willingness-to-pay parameter, or as a constrained problem maximizing $J(f)$ subject to $K(f)\le \bar{K}$.

\subsection{Finite-state existence and computation (clean baseline)}
To make the existence claim fully concrete without measure-theoretic overhead, we give the finite-state baseline.

\begin{assumption}[Finite controlled CTMC]\label{as:finite}
$\cS$ is finite. For each $x\in\cS$, $\cA(x)$ is finite. Under every stationary policy $f$, the regenerative cycle has $\E_\nu[\tau^f]\in(0,\infty)$.
\end{assumption}

\begin{theorem}[Optimal stationary policy exists in the finite model]\label{thm:finiteopt}
Under Assumption~\ref{as:finite}, for each $\lambda>0$ there exists a stationary Markov policy $f^\star$ maximizing $W_\lambda(f)=J(f)-\lambda K(f)$ over stationary Markov policies.
\end{theorem}

\begin{proof}
The set of stationary Markov policies is finite (since $\cS$ and each $\cA(x)$ are finite), so it suffices to show that $W_\lambda(f)$ is well-defined for each policy $f$. By Assumption~\ref{as:finite}, $\E_\nu[\tau^f]\in(0,\infty)$, so $J(f)$ and $K(f)$ are well-defined by Definition~\ref{def:flow}. Since $q\in[0,1]$ and $c\ge 0$, we have $J(f)\in[0,1]$ and $K(f)\in[0,\infty)$, so $W_\lambda(f)$ is finite. The maximum over a finite set of finite values is attained.
\end{proof}

\begin{remark}
In the finite model, $J(f)$ and $K(f)$ can be computed via standard linear equations for expected cycle reward and expected cycle duration, and optimization can be performed by enumeration, linear-fractional programming, or average-reward dynamic programming techniques.
\end{remark}

\subsection{General-state extensions (policy class conditions)}
For general (countable or continuous) $\cS$, standard average-reward CTMDP theory provides existence of optimal stationary policies under continuity/compactness and recurrence-type stability conditions. Those conditions are model-dependent; the core point for this foundation is that the repaired objective is an average-reward (reward-rate) criterion, so the established theory applies once stability assumptions are spelled out.

\section{Perturbation stability (sensitivity) of the repaired metric}

Practical policy evaluation requires understanding sensitivity to parameter uncertainty, which can be substantial in health economic models \citep{ultsch2015_vaccineconsensus}. A policy metric used for governance should vary continuously under small parameter changes (accident rate estimates, repair success probabilities, cost inputs).

Let $\theta$ index model parameters (transition rates, repair probabilities, etc.), and write $J_\theta(\pi)$ for the QALY flow computed in model $\theta$.

\begin{assumption}[Uniform lower bound on expected cycle length]\label{as:m}
There exists $m>0$ such that for all policies $\pi$ in a policy class $\Pi$ and parameters $\theta$ in a neighborhood of interest,
\[
\E_{\nu,\theta}[\tau^\pi]\ge m.
\]
\end{assumption}

\begin{assumption}[Lipschitz cycle-level expectations]\label{as:lipschitz}
There exist constants $L_R,L_\tau\ge 0$ and a metric $d(\cdot,\cdot)$ on parameter space such that for all $\pi\in\Pi$,
\[
\left|\E_{\nu,\theta}[R^\pi]-\E_{\nu,\theta'}[R^\pi]\right|\le L_R\, d(\theta,\theta'),
\qquad
\left|\E_{\nu,\theta}[\tau^\pi]-\E_{\nu,\theta'}[\tau^\pi]\right|\le L_\tau\, d(\theta,\theta').
\]
\end{assumption}

\begin{remark}[Sufficient conditions for Assumptions~\ref{as:m}--\ref{as:lipschitz}]
Assumption~\ref{as:m} holds if there is a uniform lower bound $\mu>0$ on the total exit rate from $\absorb$-free states: $\inf_{x\in\cS,a\in\cA(x)} Q(\absorb\mid x,a) \ge \mu$ implies $\E[\tau^\pi]\le 1/\mu$ is bounded above, and a positive catastrophe rate from some state reachable under $\nu$ gives $\E[\tau^\pi]\ge m>0$. For Assumption~\ref{as:lipschitz}, if transition rates $Q_\theta(y\mid x,a)$ are Lipschitz in $\theta$ uniformly over $x,a,y$, then standard perturbation bounds for Markov chain expectations (via Dynkin's formula or coupling arguments) yield Lipschitz continuity of $\E_\theta[R^\pi]$ and $\E_\theta[\tau^\pi]$.
\end{remark}

\begin{theorem}[Lipschitz stability of QALY flow]\label{thm:lip}
Under Assumptions~\ref{as:m}--\ref{as:lipschitz},
\[
\left|J_\theta(\pi)-J_{\theta'}(\pi)\right|
\le \frac{L_R+L_\tau}{m}\, d(\theta,\theta')
\qquad\text{for all }\pi\in\Pi.
\]
\end{theorem}

\begin{proof}
Write $J_\theta(\pi)=\frac{a}{b}$ and $J_{\theta'}(\pi)=\frac{a'}{b'}$ with
$a=\E_{\theta}[R^\pi]$, $b=\E_{\theta}[\tau^\pi]$, and similarly for $a',b'$.
We use the identity
\[
\frac{a}{b}-\frac{a'}{b'} = \frac{a b' - a' b}{b b'} = \frac{(a-a')b' + a'(b'-b)}{b b'}.
\]
Taking absolute values and applying the triangle inequality:
\[
\left|\frac{a}{b}-\frac{a'}{b'}\right|
\le \frac{|a-a'|\cdot b' + |a'|\cdot|b'-b|}{b\cdot b'}
= \frac{|a-a'|}{b} + \frac{a'}{b'}\cdot\frac{|b-b'|}{b}.
\]
Since $0\le R^\pi\le\tau^\pi$ a.s., we have $0\le a'\le b'$, so $a'/b'\le 1$. By Assumption~\ref{as:m}, $b\ge m>0$. Thus:
\[
\left|\frac{a}{b}-\frac{a'}{b'}\right|
\le \frac{|a-a'|}{m} + \frac{|b-b'|}{m}.
\]
Applying Assumption~\ref{as:lipschitz} yields the result.
\end{proof}

\begin{remark}
The stability story is explicit: (i) expected cycle length cannot collapse toward $0$, and (ii) the policy class should not create discontinuous jumps in expected cycle reward/duration under small parameter perturbations.
\end{remark}

\section{Irreversibility, risk, and volatility as explicit extensions}\label{sec:risk}

The steady flow $J(\pi)$ is the core foundation. In a maintenance world, two further dimensions commonly matter and can be added without breaking the framework: irreversibility and risk/volatility. Some health states represent irreversible conditions with distinct mortality profiles \citep{bauer_lakdawalla_reif2025_healthrisk}. The distinction between healthspan and lifespan \citep{masny2023_healthspanjustice} is naturally captured by partitioning states into reversible and absorbing subsets.

\subsection{Irreversibility as penalty or constraint}
Let $\kappa:\cS\to[0,\infty)$ encode additional disutility for irreversible states beyond their immediate quality decrement (option-value loss). Two transparent choices:

\paragraph{Penalty.}
Define adjusted quality $\tilde{q}(x)=q(x)-\kappa(x)$ (with optional clipping to keep bounds), and define $\tilde{J}(\pi)$ by replacing $q$ with $\tilde{q}$ in \eqref{eq:Rdef}--\eqref{eq:Jdef}.

\paragraph{Constraint.}
Let $\rho^\pi$ denote an irreversibility measure per cycle (e.g.\ expected time spent in $\cI$, or expected number of entries into $\cI$). Optimize $J(\pi)$ subject to $\rho^\pi\le \bar{\rho}$.

\begin{remark}
Penalty asserts a commensurable tradeoff rate; constraint asserts an acceptability boundary. Axiom~\ref{ax:explicit} demands choosing (and defending) one explicitly.
\end{remark}

\subsection{Volatility and risk sensitivity}
Average flow can hide instability. We define risk measures that are always well-defined.

\begin{assumption}[Minimum cycle length for risk measures]\label{as:mincycle}
There exists $\epsilon>0$ such that $\tau^\pi \ge \epsilon$ almost surely.
\end{assumption}

Under Assumption~\ref{as:mincycle}, the within-cycle average quality $\bar{q}^\pi \coloneqq R^\pi/\tau^\pi$ is well-defined and bounded: $\bar{q}^\pi\in[0,1]$ a.s. One can then evaluate risk-sensitive objectives such as:
\[
J_{\mathrm{MV},\lambda}(\pi)\coloneqq J(\pi)-\lambda \sqrt{\mathrm{Var}(\bar{q}^\pi)},
\]
where the variance is finite since $\bar{q}^\pi\in[0,1]$.

Alternatively, when Assumption~\ref{as:mincycle} may not hold, use risk measures applied directly to the cycle reward $R^\pi$ (which is always well-defined):
\[
J_{\mathrm{CVaR},\alpha}(\pi) \coloneqq \frac{\mathrm{CVaR}_\alpha(R^\pi)}{\E[\tau^\pi]},
\]
where $\mathrm{CVaR}_\alpha$ is the conditional value-at-risk at level $\alpha\in(0,1)$.

The parameter $\lambda$ (or $\alpha$) is an explicit normative risk-aversion coefficient.

\section{Cost-effectiveness in the steady-flow framework}

Our flow-based ICER parallels the standard incremental cost-effectiveness ratio \citep{sanders2016_secondpanel}, but replaces discounted sums with long-run averages. This avoids the well-documented issues with discount rate selection \citep{attema2018_discounting, gravelle2007_discounting}.

\begin{definition}[Flow ICER]\label{def:icer}
For two policies $\pi_1,\pi_2$ with $J(\pi_2)\neq J(\pi_1)$, define
\[
\mathrm{ICER}_{\mathrm{flow}}(\pi_1\to\pi_2)\coloneqq
\frac{K(\pi_2)-K(\pi_1)}{J(\pi_2)-J(\pi_1)}.
\]
\end{definition}

\begin{remark}
This answers a governance-relevant question: the ongoing resource cost per unit improvement in sustained quality per unit person-time.
\end{remark}

\section{Distributional aggregation (explicit ethics)}

The distributional extension follows the framework of distributional cost-effectiveness analysis \citep{asaria2013_dcea}, which assigns equity weights to different population subgroups. \citet{song_bauer_lakdawalla_reif2023_value} show that the value of health improvements depends on access to insurance and retirement products, suggesting distributional weights should account for financial as well as health heterogeneity.

Let groups be indexed by $i$ with repaired flows $J_i(\pi)$.

\paragraph{Weighted additive welfare.}
\[
SW(\pi)=\sum_i w_i\,J_i(\pi),
\]
with explicit weights $w_i$ encoding priority rules.

\paragraph{Inequality-averse welfare.}
\[
SW(\pi)=\sum_i u(J_i(\pi)),
\]
with concave $u$ encoding diminishing social marginal value.

\paragraph{Minimum-acceptable flow.}
Impose a hard constraint $J_i(\pi)\ge J_{\min}$ for all $i$ (or a bound on time spent below a quality floor). This is a normative commitment that becomes unavoidable to state clearly when lives can be arbitrarily long.

\section{Conclusion}

We proposed a foundation for health policy evaluation in an unbounded-lifespan, maintenance-driven world by taking \emph{steady person-time value} as the primary object. Under a regenerative regime interpretation, QALY flow $J(\pi)=\E[R^\pi]/\E[\tau^\pi]$ is bounded, horizon-invariant, and has a direct long-run meaning via renewal-reward theory. The same construction yields steady cost flow and a natural cost-effectiveness ratio. Stability under parameter perturbations reduces to checkable cycle-level conditions. Irreversibility and risk/volatility enter cleanly as explicit penalties or constraints, keeping all ethical commitments exposed rather than embedded in discounting.

\appendix
\section{Appendix: cross-sectional sampling interpretation}

In the regenerative process, let $U_T$ be uniform on $[0,T]$ independent of the process. 

\begin{proposition}[Cross-sectional interpretation]\label{prop:crosssection}
Assume $\E_\nu[\tau^\pi]\in(0,\infty)$. Then
\[
\E[q(X_{U_T})] \to J(\pi) \quad \text{as }T\to\infty.
\]
\end{proposition}
\begin{proof}
This is the ``inspection paradox'' form of the renewal-reward theorem. By the key renewal theorem, under $\E[\tau^\pi]<\infty$, the distribution of the state at a uniformly random time converges to the stationary distribution weighted by sojourn time, and the expected quality under this distribution equals $J(\pi)$.
\end{proof}

Thus $J(\pi)$ can be read as the expected quality of a randomly sampled person-time under the regime.

\section{Appendix: relation to discounting}

The health economics literature extensively debates discounting \citep{attema2018_discounting}. \citet{gravelle2007_discounting} analyze the decision-theoretic foundations, showing that different discount rates for costs and effects can be justified under certain conditions. \citet{omahony2015_time} examine how time enters health evaluations more broadly. Our approach sidesteps the discount rate selection problem entirely by moving to ratio-form objectives. Rather than summing discounted utilities (which requires choosing a discount rate that becomes arbitrary over unbounded horizons), we define welfare as quality per unit time, which is inherently bounded and horizon-invariant.

\bibliographystyle{apalike}
\bibliography{sources/references}

\end{document}
